\title{Parametric design of a floating offshore vertical axis wind turbine}

\documentclass[12pt]{article}

\usepackage[square,numbers,sort&compress]{natbib}
\bibliographystyle{IEEEtranN}

\usepackage{graphicx}
\graphicspath{{./}{../../gfx/}}

\usepackage{amsmath}
\usepackage{bm}
\DeclareMathOperator*{\argmin}{arg\,min}

\usepackage{booktabs}
\usepackage{tabu}
\usepackage{multirow}

\begin{document}
\maketitle


\section{Introduction}

% ---------------------------------------------------------------------------%
\subsection{Why VAWTs?}\label{sub:why_vawts_}

\citet{Arshad2013}

\citet{Henderson2010} notes that TLPs can be designed for general location, with only the mooring type specific to the site.
TLPs can also provide $10\times$ reduction in pitch and roll over spars and semi-subs.
\citet{Paquette2012} showed benefits\dots{}


% ---------------------------------------------------------------------------%
\subsection{Key issues with VAWT TLPs}\label{sub:key_issues_with_vawt_tlps}

\citet{Arshad2013}

\citet{Winterstein1994} showed that slow drift phenomena can play role in tendon loading of TLP (offshore oil \& gas).
However, per \citet{Chakrabarti2005}, first-order components are more important.
Per \citet{Chakrabarti2005}, mean offset of tendon connection angle should not exceed 5\% of water depth.
Gust are more significant offshore than onshore XX-GL.
TLPs are appropriate for water depths greater than 50\,m
\citet{Borg2014} found that a TLP VAWT could not do station keeping.
\citet{Cheng2015} found negative tensions for some DLCs.
\citet{Henderson2010} notes TLPs are potentially prone to yaw.

% ---------------------------------------------------------------------------%
\subsection{Previous work}\label{sub:previous_work}

\citet{Bachynski2012} surveyed existing TLP designs.


% ---------------------------------------------------------------------------%
\section{Methods}\label{sec:methods}

% ---------------------------------------------------------------------------%
\subsection{Design load conditions}\label{sub:design_load_conditions}

% ---------------------------------------------------------------------------%
\subsection{Parametric design}\label{sub:parametric_design}

% ---------------------------------------------------------------------------%
\subsection{Simulation}\label{sub:simulation}

The state vector is given by 

\begin{equation}
	\vec{x} = \begin{bmatrix}
	x \\
	z \\
	q \\
	\end{bmatrix}
\end{equation}

\noindent{}where the terms $x$, $z$, and $q$ represent the planar motions of surge, heave, and pitch, respectively.

The linear stiffness matrix comprises multiple components.

\begin{equation}
	\bm{K} = \bm{K^{\textrm{hs}}} + \bm{K^{\textrm{m}}}
\end{equation}

\noindent{}Here, $\bm{K^{\textrm{hs}}}$ is the hydrostatic/gravitational restoring effect and $\bm{K^{\textrm{m}}}$ is due to mooring.
The hydrostatic/gravitational restoring is a symmetric matrix of the form

\begin{equation}\label{eq:hsStiffness}
	\bm{K^{\textrm{hs}}} = \begin{bmatrix}
	0 & 0 & 0 \\
	0 & K^{\textrm{hs}}_{33} & K^{\textrm{hs}}_{35} \\
	0 & K^{\textrm{hs}}_{53} & K^{\textrm{hs}}_{55} 
	\end{bmatrix} .
\end{equation}

\noindent{}Note the elements within the definition of the hydrostatic stiffness matrix in \eqref{eq:hsStiffness} use the full six degree-of-freedom definition (i.e., 3 corresponds with heave, and 5 with pitch).
These elements can be calculated analytically from the geometry; in practice, we can take volume integrals on the discretized geometry.

\begin{equation}
	K^{\textrm{hs}}_{33} = \rho g A_{wp}
\end{equation}

\begin{equation}
	K^{\textrm{hs}}_{55} = \rho g I_{wp} + \rho g \forall z_{cob} - m g z_{cog}
\end{equation}



\citet{Bakmar2009} notes that energy rich wind has frequency range of 0-1\,Hz.

DNV-OS-J101 \cite{DNV-OS-J101} notes that wind speed distribution is a Weibull.

\citet{Jurado2018} developed a frequency domain model of floating offshore wind turbines (``QuLAF''), as did \citet{Lupton2015}.
\citet{Tracy2007} also used a frequency domain model.
\citet{Svendsen2016} developed a coupled hydrodynamic (from WAMIT) structural model for TLP wind turbines.


\begin{itemize}
	\item Aero: CACTUS
	\item Structural: OWENS
	\item Hydro
	\begin{itemize}
		\item WAMIT
		\item Nemoh
		\item Capytaine
		\item Vugts
	\end{itemize}
	\item Mooring
	\begin{itemize}
		\item \citet{Bachynski2012} used linear models from \citet{Faltinsen1998} and others
		\item \citet{AlSolihat2014}
		\item \texttt{MAP++} \cite{Masciola2014}
	\end{itemize}
	\item Cost
	\begin{itemize}
		\item \citet{Kausche2018}
		\item \citet{Bachynski2012}
		\begin{itemize}
			\item Cost is linearly proportional to displacement
		\end{itemize}
		\item \citet{Feng2010} O\&M costs for shallow water offshore winds is roughly $1.5\times{}$ that of onshore projects
		\item \citet{Arshad2013} present many different broad characterizations of cost
	\end{itemize}
\end{itemize}


\bibliography{./owt_design_refs}

\end{document}
